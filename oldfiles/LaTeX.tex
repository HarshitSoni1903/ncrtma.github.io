\documentclass[12pt]{article}
\usepackage{amsmath, amsthm}

\textwidth 12cm     
\textheight 20cm

%%% Theorem Like Envirouments

\newtheoremstyle{theorem}%name
  {10pt}		  % space above
  {10pt}  % space below
  {\sl}  % bofy font
  {\parindent}     % ident - empty=no indent,  \parindent= paragraph indent
  {\bf}  % thm head font
  {. }    % punctuation after thm head
  { }    % space after thm head: `` ``=normal \newline=linebreak
  {}     % thm head specification
\theoremstyle{theorem}
\newtheorem{theorem}{Theorem}
\newtheorem{corollary}[theorem]{Corollary}

\newtheoremstyle{defi}%name
  {10pt}		  % space above
  {10pt}  % space below
  {\rm}  % bofy font
  {\parindent}     % ident - empty=no indent,  \parindent= paragraph indent
  {\bf}  % thm head font
  {. }    % punctuation after thm head
  { }    % space after thm head: `` ``=normal \newline=linebreak
  {}     % thm head specification
\theoremstyle{defi}
\newtheorem{definition}[theorem]{Definition}

\def\proofname{\indent {\sl Proof.}}

%%%% Local Definitions start here

%%%% End of Local Definitions

\begin{document}

\title{Title of the paper}

\author{First Author$^1$ and Second Author$^2$\\
$^1$First Author's Address. The address must be writen in English\\
First Author's e-mail address\\[2pt]
$^2$Second Author's Address. The address must be writen in English\\
Second Author's e-mail address}

\maketitle

\begin{abstract}
Text of the abstract.

{\bf AMS Subject Classification:} ...

{\bf Key Words and Phrases:}...
\end{abstract}



\section{First Section of the Paper}

Text ... (see \cite{gasrah}, \cite{rosbl}, \cite{Moak})...

\begin{theorem}
Theorem text.
\end{theorem}

\begin{proof}
The proof of the theorem.
\end{proof}

\begin{corollary}
Text.
\end{corollary}

\begin{definition}
Text.
\end{definition}

\section{Second Section of the Paper}

Text ...

\begin{thebibliography}{99}

\bibitem{gasrah} G. Gasper, M. Rahman, {\it Basic Hypergeometric Series}, Cambridge University Press, Cambridge (1990).

\bibitem{rosbl} M. Rosenblum, Generalized Hermite polynomials and the Bose-like oscillator calculus, In: {\it Operator Theory: Advances and Applications}, Birkh\"auser, Basel (1994), 369-396.

\bibitem{Moak} D.S. Moak, The $q$-analogue of the Laguerre polynomials, {\it J. Math. Anal. Appl.}, {\bf 81} (1981), 20-47.

\end{thebibliography}

\end{document}
